\documentclass{article}
\usepackage{ctex}
\usepackage[a4paper,left=10mm,right=10mm,top=15mm,bottom=15mm]{geometry}
\usepackage{graphicx}
\usepackage{amsfonts,amssymb}
\usepackage{amsmath}
\usepackage{biblatex}
\usepackage{hyperref}
\usepackage{color}
\usepackage{titlesec}
\usepackage{titletoc}

\title{VI-SLAM系统初始化设计说明书(v1.0)}
\author{张谦}
\date{\today}
\begin{document}
\maketitle
\tableofcontents
\newpage

\section{问题定义}
目前闭式解BA初始化方案,还有以下几个方面问题:
\par
(1)全场景平均初始化时间(1.5s),比标杆ARCore(0.8s)慢;
\par
(2)弱纹理场景,初始化精度差;
\par
(3)小基线场景,初始化精度差。

\section{需求分析}
该初始化方案,主要有两个方面工作需要做:
\par
(1)首先需要达到快速初始化的目标:全场景平均初始化时间,达到或超越ARCore水平,同时
提升初始化的鲁棒性。在此基础上,验证弱纹理和小基线场景是否也有改善;如果没有改善,则弱纹理和
小基线问题,放到下一次优化设计中。
\par
(2)设计初始化模块的单元测试,以便初步验证该模块实现正确性,是否存在明显的错误、Crash等。

\section{方案调研}
\subsection{紧耦合闭式解初始化}

\subsection{松耦合初始化}


\section{方案设计}

\section{数学原理}

\section{流程图}

\section{类设计}

\section{子程序设计}

\section{伪代码}

\section{数据结构}

\section{调试}

\section{单元测试}

\section{效果验证}

\section{Clean Code}



\begin{thebibliography}{99}  
    \bibitem{Meyer2000}Meyer CD (2000) Matrix Analysis and Applied Linear Algebra. Philadelphia, PA: SIAM.
    \bibitem{consistency} Agostino Martinelli. Closed-form solution of visual-inertial structure from motion. International
    Journal of Computer Vision, Springer Verlag, 2013. ￿hal-00905881
\end{thebibliography}



\end{document}

